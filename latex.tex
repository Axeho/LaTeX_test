\documentclass{article}
\begin{document}

{\flushleft \large \textsf{{\LARGE{2.2}} The Schr\"odinger Equation}}\\
In 1926 and 1927, Schr\"odinger and Heisenberg published
 papers on wave mechan\-ics, descriptions of the wwave properties of
 electrons in atoms, that used very different mathematical techniques.
 In spite of the different approaches, it was soon shown that their
 theories were equivalent. Schr\"odinger's differential equations are
 more commonly used to introduce the theory, and we will follow that
 practice.\\
\fussy The Schr\"odinger equation describes the wave properties of an
 electron in terms of its position, mass, total energy, and potential
 energy. The equation is based on the \textbf{wave function}, $\Psi$,
 which describes an electron wave in space; in other words, it describes
 an atomic orbital. In its simplest notation, the equation is
\begin{displaymath}
{H\Psi~=~E\Psi}
\end{displaymath}
\begin{eqnarray}
H & = & Hamiltonian~operator    \\[6pt]
E & = & energy~of~the~electron  \\[6pt]
\Psi & = & wave~function
\end{eqnarray}
\fussy The \textbf{Hamiltonian operator}, frequently called simply the
 \emph{Hamiltonian}, includes deriva\-tives that \textbf{operate} on the
 wave function.\footnote{{An \emph{operator} is an instruction or set of
 instructions that states what to do with the function that follows it.
 It mmay be a simple instruction such as ‘‘multiply the following
 function by 6,’’ or it may be much more complicated than the
 Hamiltonian.}} When the Hamiltonian is carried out, the result is a
 constant(the energy)times $\Psi$. The operation can be performed on any
 wave function describing an atomic orbital. Different orbitals have
 different wave functions and different values of \emph{E}. This is
 another way of describing quantization in that each orbital,
 characterized by its own function $\Psi$, has a characteristic energy.\\
In the form used for calculating energy levels, the Hamiltonian operator
 for one\-electron systems is
\begin{displaymath}
{H = \frac{-h^{2}}{8\pi^{2}m}
\left(
\frac{\partial^{2}}{{\partial}x^{2}}
\frac{\partial^{2}}{{\partial}y^{2}}
\frac{\partial^{2}}{{\partial}z^{2}}
\right)}-\frac{Ze^{2}}{4\pi\varepsilon_{0}\sqrt{x^{2}+y^{2}+z^{2}}}
\end{displaymath}
\begin{center}
\begin{tabular}{p{6cm} p{5cm}}
This part of the operator describes the \emph{kinetic energy} of the
 electron, its energy of motion. & This part of the operator describes
 the \emph{potential energy} of the electron, the result of
 electrostatic attraction between the electron and the nucleus.
 It is commonly designated as \emph{V}.
\end{tabular}
\end{center}
\newpage
where\\
\begin{tabular}{r @{=} l}
$h$                            ~&~ Planck constant           \\
$m$                            ~&~ mass of the electron      \\
$e$                            ~&~ charge of the electron    \\
$\sqrt{x^{2}+y^{2}+z^{2}}~=~r$ ~&~ distance from the nucleus \\
$Z$                            ~&~ charge of the nucleus     \\
$4\pi\varepsilon_{0}$          ~&~ permittivity of a vacuum  \\
\end{tabular}\\[8pt]
This operator can be applied to a wave function $\Psi$,
\begin{displaymath}
\left[
\frac{-h^{2}}{8\pi^{2}m}
\left(
\frac{\partial^{2}}{{\partial}x^{2}}
\frac{\partial^{2}}{{\partial}y^{2}}
\frac{\partial^{2}}{{\partial}z^{2}}
\right)
+V(x, y ,z)
\right]
\Psi(x, y, z) = E\Psi(x, y, z)5
\end{displaymath}
where
\begin{displaymath}
V~=~\frac{-Ze^{2}}{4\pi\varepsilon_{0}r}~=~
\frac{-Ze^{2}}{4\pi\varepsilon_{0}\sqrt{x^{2}+y^{2}+z^{2}}}
\end{displaymath}
The potential energy $V$ is a result of electrostatic attraction between
 the electron and the nucleus. Attractive forces, such as those between
 a positivee nucleus and a negative electron, are defined by convention
 to have a negative potential energy. An electron near the nucleus
(small $r$) is strongly attracted to the nucleus and has a large
 negative potential energy. Electrons farther from the nucleus have
 potential energies that are small and negative. For an electron at
 infinite distance from the nucleus ($r~=~\infty$), the attraction
 between the nucleus and the electron is zero, and the potential energy
 is zero. The hydrogen atom energy level diagram in Figure 2.2
 illustrates these concepts.

\end{document}
